\documentclass[]{article}
\usepackage{bookman}
\usepackage{hyperref}
\usepackage[numbers]{natbib}

\hypersetup{
  % bookmarks=true,         % show bookmarks bar?
    unicode=false,          % non-Latin characters in Acrobat’s bookmarks
    pdftoolbar=true,        % show Acrobat’s toolbar?
    pdfmenubar=true,        % show Acrobat’s menu?
    pdffitwindow=false,     % window fit to page when opened
    pdfstartview={FitH},    % fits the width of the page to the window
    pdftitle={Research Statement},
    pdfauthor={Wei Liu},     % author
    pdfsubject={Wei Liu's Research Statement for Qualify},   % subject of the document
    pdfcreator={Wei Liu},   % creator of the document
    pdfproducer={Producer}, % producer of the document
    pdfkeywords={Research statement, qualify}, % list of keywords
    pdfnewwindow=true,      % links in new window
    colorlinks= true,       % false: boxed links; true: colored links
    linkcolor=red,          % color of internal links
    citecolor=green,        % color of links to bibliography
    filecolor=magenta,      % color of file links
    urlcolor=cyan           % color of external links
}

\setlength{\oddsidemargin}{0 in}
\setlength{\evensidemargin}{0 in}
\setlength{\topmargin}{-1 in}
\setlength{\textwidth}{6.5 in}
\setlength{\textheight}{9 in}
\setlength{\headsep}{0.5 in}
\setlength{\parindent}{0 in}
\setlength{\parskip}{0.1 in}

\begin{document}
\title{Research Statement} 
\author{Wei Liu\\ \small{(advisor: Tom Fletcher)} }
\date{\today} 
\maketitle 

The human brain is known to have a complicated structural network which supports
and modulates the functional connectivity. Identifying the structural and
functional connectivity helps the construction of the brain network and better
understanding of the functional integration and segregation. Various
non-invasive techniques are used to identify the anatomical and functional
connectivity of the \emph{in vivo} human brain. Among them, functional MRI
(fMRI) is widely used to identify brain's functional connectivity. The blood
oxygenation level-dependent (BOLD) signal of fMRI detects the locations of
increased neuro activity by measuring the blood oxygen levels at consecutive
time points. The connectivity is usually defined as the temporal correlation
between spatially remote regions. Compared to other imaging techniques, fMRI has
lower temporal resolution (about 2s) but higher spatial resolution (3mm).

Conventional functional MRI measures the BOLD signal change when subjects are
conducting cognitive tasks. It was found \cite{raichle2001} that there are
consistent patterns of activity at subject's resting state, where subjects
do not receive any stimulus during scan. This make it possible to find the
functional connectivity without the involvement of  any cognitive task.

The goal of my research is to characterize the functional connectivity among the
regions (voxels) of the human brain cortex by fMRI. I build models and look for
computational methods that can identify consistent connectivity among multiple
subjects. Specifically I focus on the hierarchical model with a \emph{group}
functional network that has bidirectional interaction with individual's network,
and also use Markov Random Field theory to mitigate the ambiguity of
conventional spatial smoothing.

\textbf{Past work} The highly noisy fMRI need statistical methods like
Statistical Parametric Mapping (SPM), where the effects of a stimulus signal is
estimated as a linear regression problem, with BOLD signal of stimulus as
predictor variable, and BOLD signal of any brain voxel as response
variables. Activation or no activation is decided by the significance of the
effects (i.e. regression coefficients) under the null hypothesis of no
activation. This method is often regarded as mass univariate, in the sense that
the effects of the stimulus on voxel $i$ is independent on the effect of voxel
$j$, even $i$ and $j$ are spatially adjacent. In practice, a spatial smoothing
Gaussian filter is always applied as preprocessing step, and introduces
dependence between spatially adjacent voxels' intensities (and the effects of
SPM).


While regular fMRI experiments are block design or event-related where a
stimulus is given to the subject under study, there is increasing interest in
resting-state fMRI for functional connectivity study. Because of the lack of
stimulus signal, the standard SPM method does not apply and people have new
computational methods that fall into a few
categories~\cite{margulies2010resting} listed as below.

Seed-based methods look for the linear correlation between an \emph{a priori}
region-of-interest (ROI) and all other regions (voxels) in the whole brain. This
straightforward method has the drawback of the \emph{a priori} manual selection
of ROI. Independent Component analysis (ICA) methods look for statistically
independent components without the need of selecting ROI. But users need to
manually select meaningful component by visual inspection. Clustering-based
methods partition the brain voxels into distinct regions (clusters), and voxels
in same regions belong to same functional networks. If the goal is to
discriminate the patients and healthy control groups, pattern classification
method can also be used. There are also graph theory based methods that treats
each ROI (or voxel) as a node on the graph, and the connectivity between them as
edges, and a rich set of graph algorithms can be used to learn the graph
structure (small-worldness, modularity, etc). Last there are local methods which
only look at the direct neighborhood of individual voxels.

Our first attempt~\cite{SCI:Liu2010a} on detecting functional network aims to
explicitly model the spatial smoothness of the network. Instead of applying a
Gaussian filter spatially on fMRI data as preprocessing, we use Markov Random
Field (MRF) to model the fact that if two voxels are functionally connected,
their neighbors are likely connected, too. Because the goal is to find
\emph{pairwise} connectivity among gray matter voxels, we build a graph with
twice the dimension of the original image space. A node on the graph is defined
as a pair of voxels in original image. There are links between two nodes if any
voxels in the two pairs are spatial neighbors. A mixture of Gaussian (2-class,
connected v.s. not connected) clustering is followed to identify the
connectivity variables. To learn the parameter from the data we use
\emph{Maximum a Posteriori} and \emph{Expectation-Maximization} to estimate the
connectivity as well as estimating the parameters. Because the interactions of
the spatially adjacent nodes, exactly solution is computationally intensive, so
we use mean filed theory to get the approximate solution.

This method is able to detect functional networks like default mode network. The
downside is the computation cost mostly due to the high dimensional graph,
making it difficult for generalizing to group study. Then we consider clustering
the brain voxels into non-overlapped partitions within their original space
instead of in a higher dimensional space. We notice the mean intensity and the
variance (both over all the time points) of the time course at each voxel is not
a indicator whether they belong to same functional network, so each time course
are normalized by subtracting its mean and dividing by its standard
deviation. This is equivalent to project the time courses to a high dimensional
sphere so they can be modeled by mixture of von-Mises Fisher (vMF) distribution
(similar to mixture of Gaussian on Euclidean space). The linear correlation
between two time series in original image space is equivalent to the inner
product of two points on the sphere.  MRF is again used as a spatial smoothness
prior on the hidden network labels. We estimate the network labels by maximizing
its posterior probability in a EM framework, such that voxels with same
estimated labels have larger inner product, which amounts to have larger
correlation in original space and belong to same functional network. The
introduction of MRF again poses the difficulty of computing the expectation
directly, We use Monte-Carlo Sampling to approximate the expectation value
in EM. By this method~\cite{SCI:Liu2011a} we are able to detect most significant
brain networks like motor, visual, motion, salience and executive control, and
default mode network with precision and consistency  competitive to standard ICA method.

\textbf{Current work} The high noise level of fMRI data is one main reason of
the inconsistency among individual subject's connectivity map. It is a natural
assumption that a group of subjects must share similar patterns of functional
connectivity. We are constructing a hierarchical MRF model by adding a group
network label map on top of subjects' map. It can be seen as a generative model
where individual subjects' label maps are generated given the group label map,
and time courses are generated on the sphere given the individual subjects'
labels. With this three-level model, we can estimate individual subject's label
map with group map as a prior, hence use other subjects' information during the
inference on current subject, while still allowing some degree of difference
among individuals. The group and subjects' label map are estimated iteratively
to jointly maximize the likelihood of the fMRI data. A free parameter is used to
tune the degree to which the group and individuals shared their networks. Other
parameters in MRF and mixture of vMF, can be manually set or estimated in a EM
framework. We are doing experiments to know if estimating all parameters is
appropriate or feasible.

\textbf{Future work} The data clustering and graph method are two branches of
methods to represent the brain connectivity. There is no straightforward mapping
between them. I propose that the clusters or partitions of the brain cortex
obtained from our methods can be used to define the ROI, or nodes in the
functional network's graph. The nodes should be regions with coherent functional
properties, and the definition of nodes have large influence on the
neurobiological interpretation of network topology \cite{butts2009revisiting}. A
simple parcellation scheme may group functionally different voxels into single
nodes, adding difficulties of finding consistent connections between nodes. Our
methods can be used to find a set of representative voxels from each connected
component of each functional network. Some critical issues still remain, such as
the number of clusters need to be decided.


The linear correlation as a measure of connectivity between two regions is
problematic, because it neglects the possible influence from other
variables. Two regions might have some connections as shown from their marginal
linear correlation, but actually they might be independent if the common cause
from a third region is removed. The partial correlation or conditional
dependency is  better measurement to represent the direct
connections. Another issue is when we estimate the brain network, we want a
sparse graph as a representation of the sparsity of the network. The
sparsity of the graph is consistent with the small-world properties of human
brain network: any two regions of the brain can be connected through few
intermediate nodes, yet the direct connections should be sparse
\citep{varoquaux2010brain}. In most current methods, an arbitrary thresholding
is applied on the correlation matrix to get a sparse binary network. We can
instead apply the widely used \empn{regularized linear regression} technique to
get a sparse estimation of the network. There are various methods to estimate the
conditional independence in a graph. Among them, group lasso
\cite{yuan2006model} and related methods can estimate the sparse structures of
multiple graphical models, hence can be used in group fMRI analysis.

Functional network can also be improved by integrating structural connectivity
obtained from Diffusion Tensor Imaging (DTI). It has been shown that wherever there is
an anatomical connectivity between two regions, the functional connectivity may
happen more likely. However, consistent functional connectivity is also found
between regions without direct anatomical link \citep{deco2010emerging}. This is
probably because the marginal connectivity includes both direct and indirect
links as mentioned above. This suggests that the anatomical connectivity can be used as \emph{a
  priori} knowledge when estimating the sparseness of the functional
network. The conditional independence will happen with larger probability
between regions without direct anatomical connections. It will be an interesting 
and challenging task to build a computational model to add DTI prior
in the current multi-subject graph estimation algorithms.

\bibliographystyle{plainnat}
\bibliography{/home/sci/weiliu/projects/centralref}
\end{document}
