\documentclass[]{article}
\usepackage{bookman}
\usepackage{hyperref}
\usepackage[numbers]{natbib}

\hypersetup{
  % bookmarks=true,         % show bookmarks bar?
    unicode=false,          % non-Latin characters in Acrobat’s bookmarks
    pdftoolbar=true,        % show Acrobat’s toolbar?
    pdfmenubar=true,        % show Acrobat’s menu?
    pdffitwindow=false,     % window fit to page when opened
    pdfstartview={FitH},    % fits the width of the page to the window
    pdftitle={Research Statement},
    pdfauthor={Wei Liu},     % author
    pdfsubject={Wei Liu's Research Statement for Qualify},   % subject of the document
    pdfcreator={Wei Liu},   % creator of the document
    pdfproducer={Producer}, % producer of the document
    pdfkeywords={Research statement, qualify}, % list of keywords
    pdfnewwindow=true,      % links in new window
    colorlinks= true,       % false: boxed links; true: colored links
    linkcolor=red,          % color of internal links
    citecolor=green,        % color of links to bibliography
    filecolor=magenta,      % color of file links
    urlcolor=cyan           % color of external links
}

\setlength{\oddsidemargin}{0 in}
\setlength{\evensidemargin}{0 in}
\setlength{\topmargin}{-1 in}
\setlength{\textwidth}{6.5 in}
\setlength{\textheight}{9 in}
\setlength{\headsep}{0.5 in}
\setlength{\parindent}{0 in}
\setlength{\parskip}{0.1 in}

\begin{document}
\title{Research Statement} \author{Wei Liu\\ \small{(advisor: Tom Fletcher)} }
\date{\today} \maketitle The human brain is known to have a complicated
structural network which supports and modulates the functional
connectivity. Identifying the structural and functional connectivity helps the
construction of the brain network and better understanding of the functional
integration and segregation. Various non-invasive techniques are used to
identify the anatomical and functional connectivity of the \emph{in vivo} human
brain. For anatomical connections, Diffusion Tensor Imaging (DTI) are used to
measure the white matter tracts between brain regions. For functional
connectivity, functional MRI (fMRI) is an important image modality. fMRI's blood
oxygenation level-dependent (BOLD) signal is used to detect the locations of increased
neuro activity by measuring the blood oxygen levels at consecutive time
points. The functional connectivity is usually defined as the temporal
correlation between spatially remote regions. Compared to electroencephalograph
(EEG), fMRI has lower temporal resolution (2s) but higher spatial resolution
(3mm).

Conventional functional MRI measures the BOLD signal change when subjects are
conducting cognitive tasks. It was found \cite{raichle2001} that there are consistent
patterns of activity during subject's resting state, where subjects do not
receive any stimulus during scan. This make it possible to find the functional
connectivity without any cognitive task.

The goal of my research is to characterize the functional connectivity among the
regions (voxels) of the human brain cortex by fMRI. I build models and look for 
computational methods that can identify consistent connectivity among multiple
subjects.

\textbf{Past work} The highly noisy fMRI need statistical methods like
Statistical Parametric Mapping (SPM), where the effects of a stimulus signal is
estimated as a linear regression problem, with BOLD signal of stimulus as
explanatory variable, and BOLD signal of any brain voxel as
observation. Activation or no activation is decided by the significance of the
effects under the null hypothesis of no activation. This method is often
regarded as mass univariate, in the sense that the effects of the stimulus on
voxel $i$ is independent on the effect of voxel $j$, even $i$ and $j$ are
spatially adjacent. In practice, a spatial smoothing filter is always applied as
preprocessing, thus make spatially adjacent voxels' intensities (and effects)
dependent on each other.


While regular fMRI experiments are block design or event-related where a
stimulus is given to the subject under study, there is increasing interest in
resting-state fMRI for functional connectivity study. Because of the lack of
stimulus signal, the standard SPM method does not apply and people have new
computational methods that falls into a few
categories~\cite{margulies2010resting} listed  as below.

Seed-based methods look for the linear correlation between an \emph{a priori}
region-of-interest (ROI) and all other regions (voxels) in the whole brain. This
straightforward method has the drawback of the \emph{a priori} ROI
selection. Independent Component analysis (ICA) methods look for statistically
independent components without the need of selecting ROI. But users need to
manually select meaningful component by visual inspection. Clustering-based
methods partition the brain voxels into distinct regions (clusters), and voxels
in same regions belong to same functional networks. If the goal is to
discriminate the patients and healthy control groups, pattern classification
method can also be used. There are also graph theory based methods that treats
each ROI (or voxel) as a node on the graph, and the connectivity between them as
edge, and a rich set of graph algorithms can be used to learn the graph
structure (small-worldness, modularity, etc). Last there are local methods which
only look at the direct neighborhood of individual voxels.

Our first attempt~\cite{SCI:Liu2010a} on detecting functional network aims to
explicitly model the spatial smoothness of the network. Instead of applying a
Gaussian filter spatially on fMRI data as preprocessing, we use Markov Random
Field (MRF) to model the fact that if two voxels are functionally connected,
their neighbors are likely connected, too. Because the goal is to find
\emph{pairwise} connectivity among gray matter voxels, we build a graph with
twice the dimension of the original image space. A node on the graph is defined
as a pair of voxels in original image. There are links between two nodes if any
voxels in the two pairs are spatial neighbors. A mixture of Gaussian (2-class,
connected v.s. not connected) clustering is followed to identify the
connectivity variables. To learn the parameter from the data we use
\emph{Maximum a Posteriori} and \emph{Expectation-Maximization} to estimate the
connectivity as well as estimating the parameters. Because the iterations of the
nodes, exactly solution is computationally intensive, and we use mean filed
theory to get the approximate solution.

This method is able to detect functional networks like default mode network. The
downside is the computation cost (due to the high dimensional graph) make it
difficult for generalizing to group study. Then we consider clustering the brain
voxels in their original space while still modeling spatial smoothness as
\emph{a prior} knowledge. We notice the mean intensity (over all the time
points) and the variance of the time course at each voxel is not a feature for
clustering, so each time course are normalized by subtracting its mean and
dividing by its standard deviation (also over all time points). This is
equivalent to project the time course to a high dimensional sphere and then they
can be modeled by mixture of von-Mises Fisher (vMF) distribution (similar to
mixture of Gaussian on Euclidean space). The linear correlation between two time
series in original image space is equivalent to the inner product of two points
on the sphere.  MRF is again used as a spatial smoothness prior in the model and
we estimate the network label by maximize its posterior probability in a EM
framework, such that voxels with same labels have larger inner product, hence
larger correlation in original space. Monte-Carlo Sampling is used to
approximate the expectation value in EM since the introduction of MRF again make
it difficult to compute the expectation directly. By this
method~\cite{SCI:Liu2011a} we are able to detect most significant brain networks
like motor, visual, motion, salience and control, and default mode network with
precision compared to standard ICA method.

\textbf{Current work} The noise of fMRI data make individual subject's
connectivity map not consistent. It is a natural assumption that a group of
subjects must share similar patterns of functional connectivity. We can
construct a hierarchical MRF model by adding a group label map, which we assume
should be shared by the population. In this generative model, we generate
individual subject's label map, given the group label map. And time courses are
generated on the sphere given the individual subject's labels. With this three
layer model, we can estimate the individual subject's label map with group map
as a prior, hence we also use other subjects' information during the inference
on current subject, while still allowing some degree of difference among
individuals. The parameters including one that connects the group and
individual, and those in MRF and mixture of vMF, can be manually set or
estimated in a EM framework (though we need to see if estimating all parameters
is appropriate or possible).

\textbf{Future work} The clustering and graph method are two approaches to
represent the brain connectivity, and there is no straightforward mapping
between them. The clustering or the partitioning of the brain's gray matters can
be used to define the nodes in the graph. The nodes should be regions with
coherent functional properties, and the definition of nodes have large influence
on the neurobiological interpretation of network topology
\cite{butts2009revisiting}. Some open issues still remains, such as the number
of clusters need to be decided.


The linear correlation as a measurement of connectivity
between two regions is problematic, because it neglects the possible influence
from other variables. Two regions might be independent if the comment cause from
a third regions is removed, but still will have linear correlations
marginally. The partial correlation or conditional dependency might be better
measurement to represent the direct connections. In the graphical model, the
sparsity of the graph is consistent with the small-world properties of human
brain network: any two regions of the brain can be connected through few
intermediate nodes, yet the direct connections should be sparse
\citep{varoquaux2010brain}. There are various methods to estimate the
conditional independence in a graph. Among them, group lasso
\cite{yuan2006model} and related methods can estimate the sparse structures of
multiple graphical models, hence can be used in group fMRI analysis.

Functional network can also be improved by integrating structural
connectivity obtained from DTI. It has been shown that whenever there is a anatomical
connectivity, there is functional connectivity. However, consistent functional
connectivity is also found between regions lack of direct anatomical link
\citep{deco2010emerging}. This suggest that the anatomical connectivity can be
used as a prior when estimating the sparseness of the graph. The conditional
independence will happen with bigger probability between regions where there is
no direct anatomical connections. It would be an challenging and interesting to
find a computational model to add this prior in the current multiple graphical
model estimation algorithm, such as group lasso.

\bibliographystyle{plainnat}
\bibliography{/home/sci/weiliu/projects/centralref}
\end{document}
