\documentclass[12pt]{article}
%\usepackage{bookman}
\usepackage{/home/sci/weiliu/haldefs}
\usepackage{graphicx}
\usepackage{url}
\usepackage{textcomp}
\usepackage{enumitem}
\usepackage{subfig}
\usepackage{hyperref}
%\usepackage{/home/sci/weiliu/packages/breakurl/breakurl}
\usepackage{amsmath}
\usepackage{verbatim}
\usepackage{natbib}
\usepackage{algorithmic}
\usepackage{algorithm}
\usepackage{color}
\usepackage{mdwlist}

\hypersetup{
  % bookmarks=true,         % show bookmarks bar?
    unicode=false,          % non-Latin characters in Acrobat’s bookmarks
    pdftoolbar=true,        % show Acrobat’s toolbar?
    pdfmenubar=true,        % show Acrobat’s menu?
    pdffitwindow=false,     % window fit to page when opened
    pdfstartview={FitH},    % fits the width of the page to the window
    pdftitle={Intraclass Correlation Reading Notes},
    pdfauthor={Wei Liu},     % author
    pdfsubject={Intraclass Correlation (ICC) Reading notes},   % subject of the document
    pdfcreator={Wei Liu},   % creator of the document
    pdfproducer={Producer}, % producer of the document
    pdfkeywords={Intraclass correlation coefficient, ICC}, % list of keywords
    pdfnewwindow=true,      % links in new window
    colorlinks= true,       % false: boxed links; true: colored links
    linkcolor=red,          % color of internal links
    citecolor=blue,        % color of links to bibliography
    filecolor=magenta,      % color of file links
    urlcolor=cyan           % color of external links
}


\setlength{\oddsidemargin}{0 in}
\setlength{\evensidemargin}{0 in}
\setlength{\topmargin}{-0.6 in}
\setlength{\textwidth}{6.5 in}
\setlength{\textheight}{9 in}
\setlength{\headsep}{0.5 in}
\setlength{\parindent}{0 in}
\setlength{\parskip}{0.1 in}



\begin{document}
\title{Intraclass Correlation (ICC) in fMRI Testing}
\author{Wei Liu}
\maketitle

This is a summary of the reading for the intraclass correlation (ICC) used in
fMRI. Assuming there are $N$ samples, and each sample have $K$ measurements, the
$NxK$ matrix $X$ is the data we have. the measurement can be obtained from
different raters or observers (we take raters for example in the following
text). The ICC is used to measure whether the measurement from different raters
are consistent. There are multiple ways of computing ICC depending on the
experiment condition
\cite{mcgraw1996forming,weir2005quantifying,shrout1979intraclass}. Here we take
two case of that. Before going into details of the definition, we give some
definitions of the sum of squares ($SS$) that are often used in ANOVA. These $SS$
will be used for estimation of ICC.

\begin{align*}
  x_{\cdot\cdot} &= \frac{1}{NK}\sum_{i=1}^N\sum_{j=1}^Kx_{ij}, \qquad x_{i\cdot} = \frac{1}{K}\sum_j x_{ij}, \qquad x_{\cdot j} = \frac{1}{N}\sum_i x_{ij} \\
  SS_b &= K\sum_{i = 1}^N (x_{i\cdot} - x_{\cdot\cdot})^2 \\
  SS_w &= \sum_i \sum_j (x_{ij} - x_{i\cdot})^2 \\
  SS_t &= N\sum_j (x_{\cdot j} - x_{\cdot\cdot})^2 \\
  SS_e &= \sum_i\sum_j (x_{ij} - x_{i\cdot} - x_{\cdot j} + x_{\cdot\cdot})^2 \\
  SS &= \sum_i\sum_j (x_{ij} - x_{\cdot\cdot})^2 \\
  SS &= SS_b + SS_w = SS_b + SS_t + SS_e \\
  SS_w &= SS_t + SS_e
\end{align*}
$SS$ is called total variance. by variance decomposition, it can be decomposed into a few separate parts. $SS_b$ is between-subject variance, $SS_w$ is with-in subject variance. $SS_w$ can be further decomposed into two variance (in two-way ANOVA model): $SS_t$ (between-rater variance) and $SS_e$ (residual errors).

\textbf{The first case} is similar to the one-way ANOVA model. Define $x_{ij} = \mu + r_i + w_{ij}$. $\mu$ is a fixed parameter for the population mean, $r_i$ is row effects, and are i.i.d from $\mathcal{N}(0, \sigma_b^2)$.  $w_{ij}$ is noise term and also i.i.d. from $\mathcal{N}(0, \sigma_w^2)$. The variance decomposition is $SS = SS_b + SS_w$. 

The definition of ICC in this case is \[ \rho = \frac{\sigma_b^2}{\sigma_b^2 + \sigma_w^2}.\] Because
\begin{align*}
  MS_b &= \frac{1}{N-1}SS_b, \quad \mathbb{E}[MS_b] = K\sigma_b^2 + \sigma_w^2\\
  MS_w &= \frac{1}{N(K-1)}SS_w, \quad \mathbb{E}[MS_w] = \sigma_w^2
\end{align*}
We can estimate ICC as
\[
ICC = \frac{MS_b - MS_w}{MS_b + (K-1) MS_w}
\]

\textbf{The second case} is the two-way ANOVA model where $x_{ij} = \mu + r_i + c_j +
w_{ij}$. $c_j$ is i.i.d variable in $\mathcal{N}(0, \sigma_c^2)$. We call $c_j$
a systematic error. Based on the first case, we can further decompose $SS_w =
SS_t + SS_e$. The definition of ICC with and without systematic error taken into account are\[\rho_u = \frac{\sigma_b^2}{\sigma_b^2 + \sigma_t^2 + \sigma_e^2}, \quad \rho_c = \frac{\sigma_b^2}{\sigma_b^2 + \sigma_e^2}.\] Because
\begin{align*}
    MS_b &= \frac{1}{N-1}SS_b, \quad \mathbb{E}[MS_b] = K\sigma_b^2 + \sigma_e^2\\
    MS_w &= \frac{1}{N(K-1)}SS_w, \quad \mathbb{E}[MS_w] = \sigma_c^2 + \sigma_e^2 \\
    MS_t &= \frac{1}{K-1} SS_t, \quad \mathbb{E}[MS_t] = N \sigma_c^2 + \sigma_e^2 \\
    MS_e &= \frac{1}{(N-1)(K-1)}SS_e, \quad \mathbb{E}[MS_e] = \sigma_e^2
\end{align*}
We can estimate ICC as \[ICC_u = \frac{MS_b - MS_e}{MS_b + (K-1)MS_e + (K/N)(MS_t - MS_e)}, \quad ICC_c = \frac{MS_b - MS_e}{MS_b + (K-1)MS_e}\]

Most formulas are from \cite{mcgraw1996forming}. But the definition of the sum of squares (SS) are missing. The $SS_e$ is tricky and I got the definition from \cite{zuo2010reliable}'s appendix. An example in \cite{shrout1979intraclass} confirmed the variance decomposition and $SS_e$ definition is correct.

\bibliographystyle{plainnat}
\bibliography{/home/sci/weiliu/projects/centralref}
\end{document}
