\documentclass[12pt]{article}
 
%\documentclass[fleqn]{article}
%\usepackage{palatino} 
%\usepackage{charter}
%\usepackage[T1]{fontenc}
%\usepackage{concmath} % pretty good
%\usepackage{cmbright}
%%%%%%%%%%%%%%%%%%%%%%%%%%%%%%%%%%%%%
\usepackage{/home/sci/weiliu/haldefs}
\usepackage{/home/sci/weiliu/notes}
\usepackage{/home/sci/weiliu/projects/lwdefs}
\usepackage{graphicx}
\usepackage{url}
\usepackage{textcomp}
%\usepackage[numbers]{natbib}
\usepackage{natbib}
\usepackage{subfig}
\usepackage{hyperref}
\usepackage{/home/sci/weiliu/packages/breakurl/breakurl}
%\usepackage{endfloat}
\usepackage{amsmath}
\usepackage{verbatim}
\usepackage{natbib}
\usepackage{algorithmic}
\usepackage{algorithm}

\hypersetup{
    bookmarks=true,         % show bookmarks bar?
    unicode=false,          % non-Latin characters in Acrobat’s bookmarks
    pdftoolbar=true,        % show Acrobat’s toolbar?
    pdfmenubar=true,        % show Acrobat’s menu?
    pdffitwindow=false,     % window fit to page when opened
    pdfstartview={FitH},    % fits the width of the page to the window
    pdftitle={Indepedent Component Analysis},    % title
    pdfauthor={Author},     % author
    pdfsubject={Subject},   % subject of the document
    pdfcreator={Creator},   % creator of the document
    pdfproducer={Producer}, % producer of the document
    pdfkeywords={keywords}, % list of keywords
    pdfnewwindow=true,      % links in new window
    colorlinks= true,       % false: boxed links; true: colored links
    linkcolor=red,          % color of internal links
    citecolor=green,        % color of links to bibliography
    filecolor=magenta,      % color of file links
    urlcolor=cyan           % color of external links
}



\begin{document}
\title{Notes: Indepedent Component Analysis}
\author{Wei Liu}
\maketitle
\section{sphering}
This section talks about the reason that ICA use sphering as a
preprocessing step, and if there is any relationship between this
sphering and Riemannian Manifold.

Sphering is sometimes called whitening. After sphering a zero-mean random vector $\vec x$, the components are uncorrelated and their variances equal unity.
\begin{align}
\vec z = \mat V \vec x \\
\mathbb{E}[\vec z \vec z\T] = \mat I
\end{align}

Sphering transfer the mixing matrix into a new one,
\begin{equation*}
  \vec z = \mat V \vec x = \mat V \mat A \vec s = \widetilde {\mat A} \vec s
\end{equation*}


\textbf{Reason one: } Sphering is used as processing step, because after sphering the new mixing matrix $\widetilde{\mat A}$ is orthogonal, as 
\begin{equation*}
  \mathbb{E}[\vec z \vec z\T] = \widetilde{\mat A} \mathbb{E}[\vec s \vec s\T]\widetilde{\mat A}\T = \widetilde{\mat A}\widetilde{\mat A}\T = \mat I
\end{equation*}
That means we can search the space of orthogonal matrix instead of searching in the whole space. For orthogonal matrix there are $n(n-1)/2$ parameters to estimate, which is significantly less than original $n^2$ parameters.

\bibliographystyle{plainnat}
\bibliography{/home/sci/weiliu/projects/zotero}
\end{document}
 
