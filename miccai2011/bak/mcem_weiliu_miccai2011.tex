\documentclass[runningheads,a4paper]{llncs}


\usepackage{amssymb}
\setcounter{tocdepth}{3}
\usepackage{graphicx}

\usepackage{url}
%% \urldef{\mailsa}\path|{alfred.hofmann, ursula.barth, ingrid.haas, frank.holzwarth,|
%% \urldef{\mailsb}\path|anna.kramer, leonie.kunz, christine.reiss, nicole.sator,|
%% \urldef{\mailsc}\path|erika.siebert-cole, peter.strasser, lncs}@springer.com|    
\newcommand{\keywords}[1]{\par\addvspace\baselineskip
\noindent\keywordname\enspace\ignorespaces#1}

%%%%%% Added by Wei Liu %%%%%%%
\usepackage{lwdefs}
\usepackage{amsmath}
%%%%%% End of Added by Wei Liu %%%%%%%


\begin{document}

\mainmatter  % start of an individual contribution

% first the title is needed
\title{Some Title}

% a short form should be given in case it is too long for the running head
\titlerunning{Some Title running}

% the name(s) of the author(s) follow(s) next
%
% NB: Chinese authors should write their first names(s) in front of
% their surnames. This ensures that the names appear correctly in
% the running heads and the author index.
%
\author{Some authors}
%
\authorrunning{Some author running}
% (feature abused for this document to repeat the title also on left hand pages)

% the affiliations are given next; don't give your e-mail address
% unless you accept that it will be published

%
% NB: a more complex sample for affiliations and the mapping to the
% corresponding authors can be found in the file "llncs.dem"
% (search for the string "\mainmatter" where a contribution starts).
% "llncs.dem" accompanies the document class "llncs.cls".
%

\toctitle{Lecture Notes in Computer Science}
\tocauthor{Some author}
\maketitle


\begin{abstract}
We propose a new method for detecting the functional connectivity of
the human brain in resting state. By transforming the time series data
a hyper-sphere, we build the relationship between the Pearson
correlation of time series and their shortest path on the sphere. This
makes it possible to use a mixture of von Mises-Fisher model to find
the clusters of voxels with similar pattern of activity. The model
also takes account the spatial priors by using Markov Random Field,
and we use Monte Carlo Expectation-Maximization to estimate the
cluster labels and parameters iteratively. We show how our algorithm
works for synthetic data, and on real resting-state fMRI data of 5
subjects. Our method can get obtain the clusters of connected region
on the whole brain volume, without user giving seed region of
interest.
\end{abstract}

\section{Introduction}
It has been proposed that there is a default mode network in human
brain, and functional connectivity MRI analysis is widely used to
detect the activity pattern in this network. In resting-state fMRI,
the temporal correlation between voxels are used for detecting their
connectivity. The standard procedure is to select a region of interest
(ROI) and find the correlation between the average of the ROI and
other voxels in the brain.  The correlation will usually be
thresholded so only those voxels having significant correlations with
ROI are kept.

Recent method using Independent Component Analysis to find the
components of interest, and use a ad-hoc method to manually choose
those that have specific frequency property. Like other method that
use seed regions, it does not give the full view of the brain
connectivity and tends to introduce human subjective bias.

We introduce a new method to find the pairwise connections between
voxels with a generative model. By subtracting the mean value from
each time series, and divide the length of the time series as a vector
such that it has unit second norm, we transform the time series data
in original $p+1$ dimensional space to a $p-2$ sphere. On the sphere,
if two data points have small shortest path, they will have big inner
product, and they will have big Pearson correlation in original
space. So we re-formulate the problem of finding clusters of voxels
with higher correlations, to the problem of finding clusters on sphere
that has small shortest path, or big inner products. Because von
Mises-Fisher (vMF) distribution is the counterpart on the sphere, of
the Multivariate Gaussian distribution on Euclidean space, we use vMF
to model the direction data on the sphere. Again like data in original
space, time series on spheres have multiple components, and naturally
we use mixture of vMF to model the time series data on sphere.

The next section gives introduction of von Mises-Fisher distribution
and mixture of vMF, and parameter estimate of vMF. Section
\ref{sec:mcem} introduces Monte Carlo Expectation-Maximization frame
work. We give experiments in section 5 and conclude at section 6.

\section{The von Mises-Fisher Distribution}\label{sec:vmf}
von Mises-Fisher distribution comes from the directional
statistics. It gives a distribution of data points on sphere of unit
radius. A random vector $\vec x$ with unit length ($\norm{x} = 1$ is
said to have von Mises-Fisher distribution $\cM_p(\vec \mu, \kappa)$
if its probability densitiy function is
\begin{equation}
  f(\vec x;\vec \mu, \kappa) = C_p(\kappa) \exp (\kappa \vec \mu\T \vec x),
  \quad \vec x \in S^{p-1}, 
  \label{eq:vmf}
\end{equation}
where $S^{p-1}$ is the $p$ dimensional hypersphere. $\norm{\vec \mu} =
1$ is called the \emph{mean direction}, and $\kappa \geq 0$ is called
the \emph{concentration parameter}.The larger the $\kappa$, the
greater the density is concentrated around the mean direction. Since
\eqref{eq:vmf} depends on $\vec x$ only by $\vec x\T \mu$, the vMF is
unimodal and rotationally symmetric to $\vec \mu$. The normalization
constant $C_p(\kappa)$ is given as
\begin{equation*}
  C_p(\kappa) = \frac{\kappa^{\frac{p}{2} - 1}}{(2\pi)^{\frac{p}{2}} I_{\frac{p}{2}-1}(\kappa)},
\end{equation*}
where $I_\nu$ denotes the modified Bessel function of the first kind
and order $\nu$. When $p = 2$, the vMF distribution reduces to von
Mises distribution, which gives the distribution of points on a circle
$S^1$.
\section{Monte Carlo EM for Mixture of vMF}
\label{sec:mcem}
Since we look for clusters of voxels on the sphere, we use a mixture
of vMFs. Let $\cS$ be the set of indices of all voxels and $\cL = \{1,
2, \dots, L\}$ be the set of labels. Let $\mat z$ be a configuration
of a random field $Z$ and $\cZ$ be the set of all possible
configurations so that
\begin{equation*}
  \cZ = \{ \mat z = (z_1,\dots, z_N) | z_i \in \cL, i \in \cS\}.
\end{equation*}
Each label $l\in \cL$ corresponds to a vMF component. Given $z_i = l$,
the observed data point $\vec x_i$ on sphere follows a vMF
distribution $f(\vec x_i; \vec \mu_l, \kappa_l)$. The standard EM
algorithm maximizes the expectation of log-likelihood of joint pdf of
$\mat x$ and the latent variable $\mat z$ with respect to the
posterior probability $P(\mat z | \mat x)$, i.e. $\mathbb{E}_{P(\mat
  z| \mat x)} [\log (P(\mat x, \mat z; \vec \theta))]$ Because there
is combinatorial number of configuration $\mat z$, this expectation
value is intractable. We instead use a Monte-Carlo approximation of
this expectation, defined as
\begin{equation*}
  \widetilde Q = \frac{1}{M}\sum_{m=1}^{M} \log P(\mat z^m; \vec \theta_{r}) + \log P(\mat x | \mat z^m; \vec \theta_{t}), 
\label{eq:mcemq}
\end{equation*}
where $\mat z^m$ is a sample from $P(\mat z | \mat x)$, and $\vec \theta_r$ and $\vec \theta_t$ are parameters of prior  and conditional distribution respectively.

\subsection{Markov Prior}
we have a natural assumption of the clusters of correlated voxels:
they should consist of piecewise smooth regions. We apply Markov Random
Field (MRF) on $\mat z$ to regularize this spatial smoothness. for $\mat z$ to be MRF, we define 

\begin{equation*}
  P(\mat z) = \frac{1}{C}\exp \{ -\beta \sum_{(i,j) \in \cE}T(z_i \neq z_j\}),
\end{equation*}
where $T$ is $1$ if its argument is true and $0$ otherwise. $\cE$ is
the set of voxel pairs that are spatial neighbors. $\beta$ is the only
parameter in $\vec \theta_r$, and $C$ is a normalization
constant. Because of the Markov-Gibbs equivalence, the conditional
distribution of $z_i$ at site $i$ is,
\begin{equation*}
  P(z_i | z_{-i}) = P(z_i | z_{\cN_i}) = \frac{\exp \{ -\beta \sum_{j \in \cN_i}T(z_i \neq z_j) \}}{\sum_{l \in \cL} \exp \{ -\beta \sum_{j \in \cN_i}T(l \neq z_j)\}},
\end{equation*}
where $-i$ is the set of all the sites excluding site $i$, and $\cN_{i}$ is the
set of sites of $i$'s spatial neighbors.Given the MRF priors,
\eqref{eq:mcemq} can be written as
\begin{align*}
  \widetilde Q &= \frac{1}{M}\sum_{m=1}^{M} \sum_{i \in \cS} \log P(z_i | z_{\cN_i}; \vec \theta_r) + \frac{1}{M}\sum_{m=1}^{M}\sum_{i \in \cS} \log P(\vec x_i | z_i; \vec \theta_t) \\
  &= \widetilde Q_r + \widetilde Q_t. \\
\end{align*}
We use $\widetilde Q_r$ to denote the log-likelihood of prior
distribution, and use $\widetilde Q_t$ to denote log-likelihood of
conditional distribution.

\subsection{Sampling From Posterior}
Given the parameter value $\vec \theta = \{\vec \theta_r, \vec
\theta_t \}$, we want to draw samples from the posterior distribution
$P(\mat z | \mat x; \vec \theta)$. This can be achieved by Metropolis
Sampling. Define posterior energy of $z_i$ as the negative log of the
posterior distribution $P(z_i | \vec x_i)$, the posterior energy can be
written as
\begin{equation*}
U(\mat z_i = l| \mat x_i) = \{\beta \sum_{j \in \cN_i} T(z_i \neq z_j)  \}  + \left \{ -\log(C_p(\kappa_l)) - \kappa_l \vec \mu_l \T \vec x_i \right \} + \mathrm{const}
\end{equation*}
which is the sum of prior energy and conditional energy plus a constant. Then in Metropolis Sampling, given current configuration $\mat z^n$, generate a new sample $\mat w$ by drawing a new label $l'$ at site $i$ with uniform distribution. $\mat w$ has value $l'$ at site $i$, and other sites are same as $\mat z^n$. Then we compute the change of energy $\Delta U(\mat w) = U(\mat w | \mat x) - U( \mat z^n | \mat x)$, and accept candidate $\mat w$ with probability $\min (1, \exp \{ -\Delta U(\mat w) \})$.

\section{Parameter Estimation}
By maximizing $\widetilde Q_2$ with the constraint $\vec \mu_l\T \vec \mu_l = 1$ we get 
\begin{equation}
  R_l = \sum_{m=1}^{M}\sum_{\{i \in \cS: z_i  = l\}}^{} \vec x_i, \qquad \hat {\vec \mu}_l = \frac{R_l}{\norm{R_l}}.
  \label{eq:mlmu}
\end{equation}
We do not have any \emph{a prior} knowledge or $\vec \mu$ so a Maximum Likelihood estimation  in \eqref{eq:mlmu} is best we can do. For estimating $\kappa$, we assume a Gaussian prior distribution on $\kappa$ with hyperparameters $\mu^{\kappa}$ and $\sigma^{\kappa}$, and estimate $\kappa$ by maximizing the posterior distribution. We use $\kappa$ as superscript to make distinction with vMF's mean direction vector $\vec \mu$ on the sphere.
\section{Experiments}
\bibliographystyle{splncs03} 
\bibliography{reference}
\end{document}
