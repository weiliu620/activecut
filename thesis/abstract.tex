Functional magnetic resonance imaging (fMRI) measures the change of oxygen
consumption level in the blood vessels of the human brain, hence indirectly
detecting the neuronal activity. Resting-state fMRI (rs-fMRI) is used to
identify the intrinsic functional patterns of the brain when there is no
external stimulus. Accurate estimation of intrinsic activity is important for
understanding the functional organization and dynamics of the brain, as well as
differences in the functional networks of patients with mental disorders.

This dissertation aims to robustly estimate the functional connectivities and
networks of the human brain using rs-fMRI data of multiple subjects. We use
Markov random field (MRF), an undirected graphical model to represent the
statistical dependency among the functional network variables. Graphical models
describe multivariate probability distributions that can be factorized and
represented by a graph. By defining the nodes and the edges along with their
weights according to our assumptions, we build soft constraints into the graph
structure as prior information. We explore various approximate optimization
methods including variational Bayesian, graph cuts, and Markov chain Monte Carlo
sampling (MCMC).

We develop the random field models to solve three related problems. In the first
problem, the goal is to detect the pairwise connectivity between gray matter
voxels in a rs-fMRI dataset of the single subject. We define a six-dimensional
graph to represent our prior information that two voxels are more likely to be
connected if their spatial neighbors are connected. The posterior mean of the
connectivity variables are estimated by variational inference, also known as
mean field theory in statistical physics. The proposed method proves to
outperform the standard spatial smoothing and is able to detect finer patterns
of brain activity. Our second work aims to identify multiple functional
systems. We define a Potts model, a special case of MRF, on the network label
variables, and define von Mises-Fisher distribution on the normalized fMRI
signal. The inference is significantly more difficult than the binary
classification in the previous problem. We use MCMC to draw samples from the
posterior distribution of network labels. In the third application, we extend
the graphical model to the multiple subject scenario. By building a graph
including the network labels of both a group map and the subject label maps, we
define a hierarchical model that has richer structure than the flat
single-subject model, and captures the shared patterns as well as the variation
among the subjects. All three solutions are data-driven Bayesian methods, which
estimate model parameters from the data. The experiments show that by the
regularization of MRF, the functional network maps we estimate are more accurate
and more consistent across multiple sessions.
